\documentclass[letterpaper, 10pt, conference]{ieeeconf}
\IEEEoverridecommandlockouts
\overrideIEEEmargins

\usepackage[utf8]{inputenc}

%\setlength{\textfloatsep}{0.5em}
%\setlength{\belowcaptionskip}{0.5pt}

%% utility
\usepackage{comment}
\usepackage{xspace}
\usepackage{graphicx}

\usepackage{overpic}
\usepackage{svg}
\usepackage{xcolor}
\usepackage{bm}
\usepackage[hang,flushmargin]{footmisc}
\usepackage{booktabs}
\usepackage{tabularx}

%% tables
\usepackage{tabularx}
\usepackage{multirow}
\usepackage{booktabs}
\usepackage{colortbl}
\usepackage{float}
\usepackage{placeins}

\usepackage{array}
% \newcolumntype{P}[1]{>{\centering\arraybackslash}p{#1}}  % Define centered p{} column type

% \renewcommand{\arraystretch}{1.3}

% \usepackage{etoolbox}
% \makeatletter
% \patchcmd{\@makecaption}
% {\scshape}
% {}
% {}
% {}
% \newcommand{\onetagright}{\tagsleft@false}
% \makeatother

%%  math
\let\proof\relax
\let\endproof\relax
\usepackage{amsmath}
\usepackage{amssymb}
\usepackage{amsthm}
\usepackage{siunitx}
%\usepackage{bm}

%% theorems
% \renewcommand\qedsymbol{\newline$\square$}
% \newtheoremstyle{main}
% {1em}                                                % space above
% {1em}                                                % space below
% {\itshape}                                        % bodyfont
% {0pt}                                                % indent
% {\scshape}                                           % head font
% {\\*}                                                % head punctuation
% {2pt}                                                % head space
% {\thmname{#1}\thmnumber{ #2}: \thmnote{\itshape #3}} % head spec

%\theoremstyle{main}
\newtheorem{definition}{Definition}[section]
\newtheorem{subdefinition}{Definition}[definition]
\newtheorem{theorem}{Theorem}[section]
\newtheorem{lemma}{Lemma}[section]
\newtheorem{proposition}{Proposition}[section]
\newtheorem{claim}{Claim}[section]
\newtheorem{assumption}{Assumption}[]
\newtheorem{subassumption}{Assumption}[assumption]
\renewcommand{\thesubassumption}{\theassumption\alph{subassumption}}

\usepackage{environ}

%% algorithms
\usepackage{algorithm}
\usepackage{algpseudocode}

% \newcommand{\removelatexerror}{\let\@latex@error\@gobble}

%% microtype
\usepackage[
	activate   = {true},
	protrusion = false,
	expansion  = true,
	kerning    = true,
	spacing    = true,
	tracking   = false,
	auto       = true,
	selected   = true,
	factor     = 2000,
	stretch    = 50,
	shrink     = 20,
]{microtype}

%% biblatex
\usepackage{csquotes}
\usepackage[
	maxbibnames=99,
	maxcitenames=2,
	natbib=true,
	style=numeric-comp,
	backend=biber,
	sorting=none,
	giveninits=true,
	url=false,
	doi=false,
	eprint=false,
	isbn=false,
]{biblatex}

\addbibresource{references.bib}
\renewcommand*{\bibfont}{\footnotesize}
\definecolor{gg}{RGB}{240, 74, 0}

%% hyperref
\makeatletter
\let\NAT@parse\undefined
\makeatother
\usepackage[pdfa,colorlinks,bookmarksopen,bookmarksnumbered,allcolors=gg]{hyperref}
% \usepackage{cite}
\usepackage[english]{babel}
%% autoref
\usepackage[nameinlink,capitalise]{cleveref}
\crefname{line}{line}{lines}
\crefname{figure}{Fig.}{Figs.}
\Crefname{figure}{Fig.}{Figs.}
\crefname{equation}{Eq.}{Eqs.}
\Crefname{equation}{Eq.}{Eqs.}
\crefname{section}{Sec.}{Secs.}
\Crefname{section}{Sec.}{Secs.}
\crefname{definition}{Def.}{Defs.}
\Crefname{definition}{Def.}{Defs.}
\crefname{algorithm}{Alg.}{Algs.}
\Crefname{algorithm}{Alg.}{Algs.}
\crefname{assumption}{Asm.}{Asms.}
\Crefname{assumption}{Asm.}{Asms.}
\crefname{subassumption}{Asm.}{Asms.}
\Crefname{subassumption}{Asm.}{Asms.}
\Crefname{problem}{Problem}{Problems}
\crefname{problem}{Problem}{Problems}


\newcommand{\mf}[1]{\mbox{\cref{#1}}\xspace}
\newcommand{\mfa}[1]{\mbox{\cref{#1}}}

\usepackage{flushend}

\let\labelindent\relax
\usepackage[inline]{enumitem}
\usepackage{mathtools}

%% macros
%% custom commands for editing
\newcommand{\todo}[1]{{\color{red}TODO: #1}}
\newcommand{\info}[1]{{\color{blue}INFO: #1}}

\newcommand{\ompl}{\textsc{ompl}\xspace}
\newcommand{\cspace}{\mbox{\ensuremath{\mathcal{Q}}}\xspace}
\newcommand{\cspacespace}{\mbox{\ensuremath{\mathcal{Q}}}-space\xspace}
\newcommand{\cgoal}{\mbox{\ensuremath{\mathcal{Q}_{\rm goal}}}\xspace}
\newcommand{\cfree}{\mbox{\ensuremath{\mathcal{Q}_{\rm free}}}\xspace}
\newcommand{\cstart}{\mbox{\ensuremath{q_{start}}}\xspace}
\newcommand{\workspace}{\mbox{\ensuremath{\mathcal{W}}}\xspace}

\newcommand{\se}{\ensuremath{\textsc{se}(\oldstylenums{3})}\xspace}
\newcommand{\dof}{\textsc{d\scalebox{.8}{o}f}\xspace}
\newcommand{\psprm}{\textsc{ps-prm}\xspace}
\newcommand{\ik}{\textsc{ik}\xspace}
\newcommand{\eg}{\emph{e.g.},\xspace}
\newcommand{\ie}{\emph{i.e.},\xspace}

% Calligraphic fonts
\newcommand{\calA}{{\cal A}}
\newcommand{\calB}{{\cal B}}
\newcommand{\calC}{{\cal C}}
\newcommand{\calD}{{\cal D}}
\newcommand{\calE}{{\cal E}}
\newcommand{\calF}{{\cal F}}
\newcommand{\calG}{{\cal G}}
\newcommand{\calH}{{\cal H}}
\newcommand{\calI}{{\cal I}}
\newcommand{\calJ}{{\cal J}}
\newcommand{\calK}{{\cal K}}
\newcommand{\calL}{{\cal L}}
\newcommand{\calM}{{\cal M}}
\newcommand{\calN}{{\cal N}}
\newcommand{\calO}{{\cal O}}
\newcommand{\calP}{{\cal P}}
\newcommand{\calQ}{{\cal Q}}
\newcommand{\calR}{{\cal R}}
\newcommand{\calS}{{\cal S}}
\newcommand{\calT}{{\cal T}}
\newcommand{\calU}{{\cal U}}
\newcommand{\calV}{{\cal V}}
\newcommand{\calW}{{\cal W}}
\newcommand{\calX}{{\cal X}}
\newcommand{\calY}{{\cal Y}}
\newcommand{\calZ}{{\cal Z}}

% Fraktur fonts
\newcommand{\frakA}{{\mathfrak{A}}}
\newcommand{\frakB}{{\mathfrak{B}}}
\newcommand{\frakC}{{\mathfrak{C}}}
\newcommand{\frakD}{{\mathfrak{D}}}
\newcommand{\frakE}{{\mathfrak{E}}}
\newcommand{\frakF}{{\mathfrak{F}}}
\newcommand{\frakG}{{\mathfrak{G}}}
\newcommand{\frakH}{{\mathfrak{H}}}
\newcommand{\frakI}{{\mathfrak{I}}}
\newcommand{\frakJ}{{\mathfrak{J}}}
\newcommand{\frakK}{{\mathfrak{K}}}
\newcommand{\frakL}{{\mathfrak{L}}}
\newcommand{\frakM}{{\mathfrak{M}}}
\newcommand{\frakN}{{\mathfrak{N}}}
\newcommand{\frakO}{{\mathfrak{O}}}
\newcommand{\frakP}{{\mathfrak{P}}}
\newcommand{\frakQ}{{\mathfrak{Q}}}
\newcommand{\frakR}{{\mathfrak{R}}}
\newcommand{\frakS}{{\mathfrak{S}}}
\newcommand{\frakT}{{\mathfrak{T}}}
\newcommand{\frakU}{{\mathfrak{U}}}
\newcommand{\frakV}{{\mathfrak{V}}}
\newcommand{\frakW}{{\mathfrak{W}}}
\newcommand{\frakX}{{\mathfrak{X}}}
\newcommand{\frakY}{{\mathfrak{Y}}}
\newcommand{\frakZ}{{\mathfrak{Z}}}

\newcommand{\fraka}{{\mathfrak{a}}}
\newcommand{\frakb}{{\mathfrak{b}}}
\newcommand{\frakc}{{\mathfrak{c}}}
\newcommand{\frakd}{{\mathfrak{d}}}
\newcommand{\frake}{{\mathfrak{e}}}
\newcommand{\frakf}{{\mathfrak{f}}}
\newcommand{\frakg}{{\mathfrak{g}}}
\newcommand{\frakh}{{\mathfrak{h}}}
\newcommand{\fraki}{{\mathfrak{i}}}
\newcommand{\frakj}{{\mathfrak{j}}}
\newcommand{\frakk}{{\mathfrak{k}}}
\newcommand{\frakl}{{\mathfrak{l}}}
\newcommand{\frakm}{{\mathfrak{m}}}
\newcommand{\frakn}{{\mathfrak{n}}}
\newcommand{\frako}{{\mathfrak{o}}}
\newcommand{\frakp}{{\mathfrak{p}}}
\newcommand{\frakq}{{\mathfrak{q}}}
\newcommand{\frakr}{{\mathfrak{r}}}
\newcommand{\fraks}{{\mathfrak{s}}}
\newcommand{\frakt}{{\mathfrak{t}}}
\newcommand{\fraku}{{\mathfrak{u}}}
\newcommand{\frakv}{{\mathfrak{v}}}
\newcommand{\frakw}{{\mathfrak{w}}}
\newcommand{\frakx}{{\mathfrak{x}}}
\newcommand{\fraky}{{\mathfrak{y}}}
\newcommand{\frakz}{{\mathfrak{z}}}

\newcommand{\bffraka}{{\boldsymbol{\mathfrak{a}}}}
\newcommand{\bffrakb}{{\boldsymbol{\mathfrak{b}}}}
\newcommand{\bffrakc}{{\boldsymbol{\mathfrak{c}}}}
\newcommand{\bffrakd}{{\boldsymbol{\mathfrak{d}}}}
\newcommand{\bffrake}{{\boldsymbol{\mathfrak{e}}}}
\newcommand{\bffrakf}{{\boldsymbol{\mathfrak{f}}}}
\newcommand{\bffrakg}{{\boldsymbol{\mathfrak{g}}}}
\newcommand{\bffrakh}{{\boldsymbol{\mathfrak{h}}}}
\newcommand{\bffraki}{{\boldsymbol{\mathfrak{i}}}}
\newcommand{\bffrakj}{{\boldsymbol{\mathfrak{j}}}}
\newcommand{\bffrakk}{{\boldsymbol{\mathfrak{k}}}}
\newcommand{\bffrakl}{{\boldsymbol{\mathfrak{l}}}}
\newcommand{\bffrakm}{{\boldsymbol{\mathfrak{m}}}}
\newcommand{\bffrakn}{{\boldsymbol{\mathfrak{n}}}}
\newcommand{\bffrako}{{\boldsymbol{\mathfrak{o}}}}
\newcommand{\bffrakp}{{\boldsymbol{\mathfrak{p}}}}
\newcommand{\bffrakq}{{\boldsymbol{\mathfrak{q}}}}
\newcommand{\bffrakr}{{\boldsymbol{\mathfrak{r}}}}
\newcommand{\bffraks}{{\boldsymbol{\mathfrak{s}}}}
\newcommand{\bffrakt}{{\boldsymbol{\mathfrak{t}}}}
\newcommand{\bffraku}{{\boldsymbol{\mathfrak{u}}}}
\newcommand{\bffrakv}{{\boldsymbol{\mathfrak{v}}}}
\newcommand{\bffrakw}{{\boldsymbol{\mathfrak{w}}}}
\newcommand{\bffrakx}{{\boldsymbol{\mathfrak{x}}}}
\newcommand{\bffraky}{{\boldsymbol{\mathfrak{y}}}}
\newcommand{\bffrakz}{{\boldsymbol{\mathfrak{z}}}}

% Sets:
\newcommand{\setA}{\textsf{A}}
\newcommand{\setB}{\textsf{B}}
\newcommand{\setC}{\textsf{C}}
\newcommand{\setD}{\textsf{D}}
\newcommand{\setE}{\textsf{E}}
\newcommand{\setF}{\textsf{F}}
\newcommand{\setG}{\textsf{G}}
\newcommand{\setH}{\textsf{H}}
\newcommand{\setI}{\textsf{I}}
\newcommand{\setJ}{\textsf{J}}
\newcommand{\setK}{\textsf{K}}
\newcommand{\setL}{\textsf{L}}
\newcommand{\setM}{\textsf{M}}
\newcommand{\setN}{\textsf{N}}
\newcommand{\setO}{\textsf{O}}
\newcommand{\setP}{\textsf{P}}
\newcommand{\setQ}{\textsf{Q}}
\newcommand{\setR}{\textsf{R}}
\newcommand{\setS}{\textsf{S}}
\newcommand{\setT}{\textsf{T}}
\newcommand{\setU}{\textsf{U}}
\newcommand{\setV}{\textsf{V}}
\newcommand{\setW}{\textsf{W}}
\newcommand{\setX}{\textsf{X}}
\newcommand{\setY}{\textsf{Y}}
\newcommand{\setZ}{\textsf{Z}}

% Vectors
\newcommand{\bfa}{\mathbf{a}}
\newcommand{\bfb}{\mathbf{b}}
\newcommand{\bfc}{\mathbf{c}}
\newcommand{\bfd}{\mathbf{d}}
\newcommand{\bfe}{\mathbf{e}}
\newcommand{\bff}{\mathbf{f}}
\newcommand{\bfg}{\mathbf{g}}
\newcommand{\bfh}{\mathbf{h}}
\newcommand{\bfi}{\mathbf{i}}
\newcommand{\bfj}{\mathbf{j}}
\newcommand{\bfk}{\mathbf{k}}
\newcommand{\bfl}{\mathbf{l}}
\newcommand{\bfm}{\mathbf{m}}
\newcommand{\bfn}{\mathbf{n}}
\newcommand{\bfo}{\mathbf{o}}
\newcommand{\bfp}{\mathbf{p}}
\newcommand{\bfq}{\mathbf{q}}
\newcommand{\bfr}{\mathbf{r}}
\newcommand{\bfs}{\mathbf{s}}
\newcommand{\bft}{\mathbf{t}}
\newcommand{\bfu}{\mathbf{u}}
\newcommand{\bfv}{\mathbf{v}}
\newcommand{\bfw}{\mathbf{w}}
\newcommand{\bfx}{\mathbf{x}}
\newcommand{\bfy}{\mathbf{y}}
\newcommand{\bfz}{\mathbf{z}}


\newcommand{\bfalpha}{\boldsymbol{\alpha}}
\newcommand{\bfbeta}{\boldsymbol{\beta}}
\newcommand{\bfgamma}{\boldsymbol{\gamma}}
\newcommand{\bfdelta}{\boldsymbol{\delta}}
\newcommand{\bfepsilon}{\boldsymbol{\epsilon}}
\newcommand{\bfzeta}{\boldsymbol{\zeta}}
\newcommand{\bfeta}{\boldsymbol{\eta}}
\newcommand{\bftheta}{\boldsymbol{\theta}}
\newcommand{\bfiota}{\boldsymbol{\iota}}
\newcommand{\bfkappa}{\boldsymbol{\kappa}}
\newcommand{\bflambda}{\boldsymbol{\lambda}}
\newcommand{\bfmu}{\boldsymbol{\mu}}
\newcommand{\bfnu}{\boldsymbol{\nu}}
\newcommand{\bfomicron}{\boldsymbol{\omicron}}
\newcommand{\bfpi}{\boldsymbol{\pi}}
\newcommand{\bfrho}{\boldsymbol{\rho}}
\newcommand{\bfsigma}{\boldsymbol{\sigma}}
\newcommand{\bftau}{\boldsymbol{\tau}}
\newcommand{\bfupsilon}{\boldsymbol{\upsilon}}
\newcommand{\bfphi}{\boldsymbol{\phi}}
\newcommand{\bfchi}{\boldsymbol{\chi}}
\newcommand{\bfpsi}{\boldsymbol{\psi}}
\newcommand{\bfomega}{\boldsymbol{\omega}}
\newcommand{\bfxi}{\boldsymbol{\xi}}
\newcommand{\bfell}{\boldsymbol{\ell}}

% Matrices
\newcommand{\bfA}{\mathbf{A}}
\newcommand{\bfB}{\mathbf{B}}
\newcommand{\bfC}{\mathbf{C}}
\newcommand{\bfD}{\mathbf{D}}
\newcommand{\bfE}{\mathbf{E}}
\newcommand{\bfF}{\mathbf{F}}
\newcommand{\bfG}{\mathbf{G}}
\newcommand{\bfH}{\mathbf{H}}
\newcommand{\bfI}{\mathbf{I}}
\newcommand{\bfJ}{\mathbf{J}}
\newcommand{\bfK}{\mathbf{K}}
\newcommand{\bfL}{\mathbf{L}}
\newcommand{\bfM}{\mathbf{M}}
\newcommand{\bfN}{\mathbf{N}}
\newcommand{\bfO}{\mathbf{O}}
\newcommand{\bfP}{\mathbf{P}}
\newcommand{\bfQ}{\mathbf{Q}}
\newcommand{\bfR}{\mathbf{R}}
\newcommand{\bfS}{\mathbf{S}}
\newcommand{\bfT}{\mathbf{T}}
\newcommand{\bfU}{\mathbf{U}}
\newcommand{\bfV}{\mathbf{V}}
\newcommand{\bfW}{\mathbf{W}}
\newcommand{\bfX}{\mathbf{X}}
\newcommand{\bfY}{\mathbf{Y}}
\newcommand{\bfZ}{\mathbf{Z}}


\newcommand{\bfGamma}{\boldsymbol{\Gamma}}
\newcommand{\bfDelta}{\boldsymbol{\Delta}}
\newcommand{\bfTheta}{\boldsymbol{\Theta}}
\newcommand{\bfLambda}{\boldsymbol{\Lambda}}
\newcommand{\bfPi}{\boldsymbol{\Pi}}
\newcommand{\bfSigma}{\boldsymbol{\Sigma}}
\newcommand{\bfUpsilon}{\boldsymbol{\Upsilon}}
\newcommand{\bfPhi}{\boldsymbol{\Phi}}
\newcommand{\bfPsi}{\boldsymbol{\Psi}}
\newcommand{\bfOmega}{\boldsymbol{\Omega}}


% Blackboard Bold:
\newcommand{\bbA}{\mathbb{A}}
\newcommand{\bbB}{\mathbb{B}}
\newcommand{\bbC}{\mathbb{C}}
\newcommand{\bbD}{\mathbb{D}}
\newcommand{\bbE}{\mathbb{E}}
\newcommand{\bbF}{\mathbb{F}}
\newcommand{\bbG}{\mathbb{G}}
\newcommand{\bbH}{\mathbb{H}}
\newcommand{\bbI}{\mathbb{I}}
\newcommand{\bbJ}{\mathbb{J}}
\newcommand{\bbK}{\mathbb{K}}
\newcommand{\bbL}{\mathbb{L}}
\newcommand{\bbM}{\mathbb{M}}
\newcommand{\bbN}{\mathbb{N}}
\newcommand{\bbO}{\mathbb{O}}
\newcommand{\bbP}{\mathbb{P}}
\newcommand{\bbQ}{\mathbb{Q}}
\newcommand{\bbR}{\mathbb{R}}
\newcommand{\bbS}{\mathbb{S}}
\newcommand{\bbT}{\mathbb{T}}
\newcommand{\bbU}{\mathbb{U}}
\newcommand{\bbV}{\mathbb{V}}
\newcommand{\bbW}{\mathbb{W}}
\newcommand{\bbX}{\mathbb{X}}
\newcommand{\bbY}{\mathbb{Y}}
\newcommand{\bbZ}{\mathbb{Z}}

%
% \newcommand{\scFK}{\textsc{FK}}
% \newcommand{\scCC}{\textsc{CC}}
% \newcommand{\scNN}{\textsc{NN}}
% \newcommand{\scSIMD}{\textsc{SIMD}}

\newcommand{\ourplanner}{\textsc{GANC-PRM}\xspace}

\graphicspath{ {./images/} }
\pdfminorversion=4


\title{\fontsize{17pt}{24pt}\selectfont \bf
GANC-PRM: GPU-Accelerated PRM with Neural Costs
}
\author{Anonymous Author(s)}
% \author{Leo Bashaw $^{1,*}$, Qingxi Meng $^{2,*}$, Emiliano Flores $^{2}$, Weihang Guo $^{2}$, and Lydia E. Kavraki$^{2,3}$% <-this % stops a space
% \thanks{
% $^{1}$ Department of Electrical and Computer Engineering, Rice University\newline
% $^{2}$ Department of Computer Science, Rice University\newline 
% $^{3}$  Ken Kennedy Institute, Rice University\newline
% {\tt \{lb73, qm15, ef55, wg25, kavraki\}@rice.edu}\newline
% $^{*}$ Equal contribution.\newline 
% }%   
% }

\newcommand{\Weihang}[1]{{\color{blue}[Weihang: #1]}}
\newcommand{\Emiliano}[1]{{\color{purple}[Emiliano: #1]}}
\newcommand{\Qingxi}[1]{{\color{red}[Qingxi: #1]}}


\begin{document}
\maketitle
\thispagestyle{empty}
\pagestyle{empty}

\begin{abstract}
The sampling-based motion planning algorithm Probabilistic Roadmap (PRM) is widely used to efficiently solve high-dimensional planning problems. Classical PRM focuses on collision avoidance, obstacle clearance, and path smoothness, yet many real-world applications require planners to also reason about the broader context and task in which they are deployed. For example, in human–robot collaboration (HRC), a robot must generate smooth collision-free paths while also accounting for human intentions, safety, or privacy. Such context-aware objectives can be represented by neural cost functions, which learn task-relevant representations of the environment and are naturally optimized for execution on a GPU. The incorporation of neural cost functions into traditional CPU-based implementations of PRM could enable context-aware planning, but the resulting CPU-GPU communication overhead would prohibit planning online. GPU-based implementations of PRM have been published, yet none leverage neural cost functions. To address this gap, we present \ourplanner, a GPU-accelerated PRM framework that incorporates neural cost functions to generate solutions optimized for a specific planning context. We demonstrate the real-time context-aware planning capability of \ourplanner in both simulated and real-world robotic experiments. We sustain planning frequencies of up to 75~Hz and achieve up to a 130$\times$ speedup over a baseline CPU implementation.
 
% Incorporating such context-aware objectives into PRM is difficult because the corresponding cost functions are often computationally expensive to evaluate during roadmap construction. Many of these objectives can be expressed as neural cost functions, which learn task-relevant representations of the environment and are naturally optimized for GPU execution. Integrating neural functions into conventional CPU-based PRM implementations, however, introduces substantial CPU–GPU communication overhead. Prior work has investigated hardware-accelerated variants of PRM, yet no existing approach fully integrates neural costs within a GPU-native planner. 

% The sampling-based motion planning algorithm Probabilistic Roadmap (PRM) offers efficient solutions for high-dimensional planning problems encountered by real-world robots. However, some applications require planning under user-defined cost functions, which are often represented by neural networks. Incorporating these complex cost functions significantly increases computational demands—particularly because traditional PRM implementations run on the CPU, while neural models are optimized for execution on the GPU. Although previous work has explored parallelizing components of PRM, no existing approach integrates neural cost functions into PRM within a fully parallelized framework. In this work, we present \ourplanner, a GPU-accelerated implementation of PRM that achieves parallelism across the entire algorithm while incorporating neural cost functions. We evaluate the effectiveness of \ourplanner in a variety of simulation and real-world robotic experiments, demonstrating significantly improved performance compared to baseline methods.
\end{abstract}

\section{Introduction}
\label{sec:introduction}

Motion planning algorithms~\cite{kavraki2002probabilistic, kuffner2000rrt} are a crucial component of autonomous robotic systems, as they compute collision-free paths in a robot’s configuration space and thereby enable safe navigation in unfamiliar environments~\cite{latombe2012robot}. For robots operating in dynamic settings, such planners must generate motion plans multiple times per second to react to sudden environmental changes. Among many approaches, sampling-based motion planning (SBMP)~\cite{tamizi2023review, zhou2022review} has emerged as a successful category of planning algorithms, particularly well-suited for solving high-dimensional problems where exact methods become intractable. Within this category, the probabilistic roadmap (PRM)~\cite{kavraki2002probabilistic} is a widely used method that constructs a graph of sampled configurations and then searches this graph to identify paths between a given start and goal.


\begin{figure}[!t]
    \includegraphics[width=1.0\columnwidth]{image/rr_thrown_cropped.pdf}
    \caption{
    \ourplanner allows the robot to replan at up to \textbf{75~Hz} while it monitors and pursues an object (the red oval) during a game of ``keep-away" with two humans. As the humans throw the object back and forth, the robot tracks and pursues it. The robot's path and the camera's optical axis are shown in green and blue, respectively. The object's trajectory is shown in orange.
    }
    \label{fig:real_robot_thrown}
\end{figure}


Classical PRM focuses on collision avoidance and path properties such as obstacle clearance and smoothness, yet many real-world applications require that planners also reason about the broader context and task in which they are deployed. For example, in human–robot interaction (HRI) and human–robot collaboration (HRC), a robot must generate smooth, collision-free paths while also accounting for factors such as human intentions, safety, and privacy~\cite{mainprice2011planning, rajendran2021human}. Incorporating such context-aware objectives into PRM is difficult because the corresponding cost functions are often computationally expensive to evaluate during roadmap construction. Many of these objectives can be expressed as neural cost functions~\cite{zhao2022perception, chen2025int2planner, you2025human}, which learn task-relevant representations of the environment and are naturally optimized for GPU execution. Therefore, integrating neural costs into conventional CPU-based PRM implementations introduces substantial CPU–GPU communication overhead. Prior work has introduced hardware-accelerated variants of PRM that achieve real-time performance~\cite{blankenburg2020towards,pan2010g,pan2010efficient,pan2012gpu,amato1999probabilistic}, yet none integrate neural costs within a GPU-native planner.


% A common method of enabling real-time motion planning is the use of various hardware-acceleration strategies ~\cite{thomason2024motions, ramsey2024collision, wilson2025nearest, sundaralingam2023curobo, lee2024gpu, park2013real, pan2012gpu, atay2006motion, murray2016microarchitecture, huang2022hardware}, such as single-instruction multiple-thread~(SIMT)~\cite{sundaralingam2023curobo, lee2024gpu, park2013real, pan2012gpu} or single-instruction multiple-data~(SIMD)~\cite{thomason2024motions, ramsey2024collision, wilson2025nearest} parallelism.   Single-threaded, CPU-based implementations of PRM are slow; however, there exist several hardware-accelerated versions of PRM that achieve real-time performance~\cite{blankenburg2020towards,pan2010g,pan2010efficient,pan2012gpu,amato1999probabilistic}.


% Beyond the real-time planning requirement, algorithms deployed on autonomous robots often require awareness of the specific task they are meant to execute. These tasks may cover human-robot interaction (HRI)~\cite{mainprice2011planning, rajendran2021human}, visibility and perception of objects in the environment ~\cite{falanga2018pampc, tordesillas2023deep, meng2025look} or safety during movement. A popular method of transferring this context awareness to planning is the use of neural cost functions~\cite{saha2024edmp, huang2023differentiable, yang2021real}, which learn features from data rather than requiring a manually defined function.

% should ideally accommodate planning under various constrained settings such as frozen joints, perception objectives, privacy centric settings, or collaboration with humans.  A popular method of capturing these constraints is the use of neural cost functions~\cite{saha2024edmp, huang2023differentiable, yang2021real}, which learn rich features from data rather than requiring a human to manually define a cost function.  
% One disadvantage of the fully learning-based planning approach is that it requires extensive training (which itself requires large amounts of data and compute resources), is not highly generalizable beyond the scope of its training data, and does not offer the same probabilistic completeness guarantees as SBMPs~\cite{noroozi2023conventional}. 
% Moreover, many constrained planning algorithms are tailored to their specific applications and do not provide a framework designed to adapt to a combination of hand-crafted and neural cost functions ~\cite{liu2023task, sisbot2012human, xi2024lightweight, papachristos2019localization,bartolomei2020perception,sundaralingam2023curobo,hu2025cprrtc}.


To address this gap, we propose \ourplanner, a \textbf{G}PU-\textbf{A}ccelerated PRM with \textbf{N}eural \textbf{C}osts, which integrates neural cost functions to generate solutions tailored to specific planning contexts. \ourplanner builds on PRM’s inherent parallelism and its ability to globally sample the configuration space, enabling the planner to optimize paths with respect to neural costs and traditional PRM costs. We focus on PRMs rather than tree-based SBMPs (e.g.,~\cite{lavalle1998rapidly}) because PRMs construct global representations of the configuration space rather than committing to the first feasible solution. This perspective aligns naturally with neural costs, which also capture global structure of the configuration space. In \ourplanner, neural costs assign context-aware scores to each node in the roadmap, and these scores are subsequently incorporated into graph search to bias solutions toward context-aware objectives.


We demonstrate the effectiveness of \ourplanner in a representative context-aware setting, namely perception-aware motion planning. In this case, \ourplanner achieves real-time planning frequencies of up to 75~Hz while simultaneously satisfying perception-aware constraints encoded by a neural cost function. We evaluate our planner on Stretch 2 from Hello Robot~\cite{kemp2022design} in both simulated and real environments and compare its performance to a baseline CPU implementation from~\cite{meng2025look}. Our results demonstrate that~\ourplanner framework can be deployed on real robots to enable context-aware real-time motion planning.

% producing paths that are jointly optimized for both traditional motion costs and context-aware objectives.

\begin{figure*}[!ht]
    \centering
    \includegraphics[width=0.8\textwidth]{image/high_res_human_chair.pdf}
    \caption{
    In this figure, the robot must traverse the room while continuously maintaining a frontal view of the human. Initially, as the human faces right, the robot plans a path along the right side (red arrow). When the human turns to face left, the robot replans and switches to a left-side path (blue arrow). Finally, when the human moves the white chair, the robot replans once more in real time to avoid the new obstacle (green arrow).
    }
    \label{fig:real_robot_dynamic}
\end{figure*}

% In this work, we make three key contributions:
% \begin{itemize}
% \item  We propose a GPU-accelerated implementation of PRM that achieves parallelism across the entire algorithm while incorporating neural cost functions.
% \item Through extensive simulated and real-robot experiments, we demonstrate that our planner outperforms the best baseline method.
% \end{itemize}


% \begin{itemize}

% \item  Need motion plans that optimize paths for user-defined cost functions. Why? We want to model and optimize solutions for the abstract qualities of a path, such as the visibility of the faces of actors in the environment.

% \item Need the planning pipeline to run in real time so that motion plans can be regenerated on the fly as actors engage with the environment.

% \item Real-time requirement necessitates GPU-based PRM for full integration with the neural network representations of the cost functions. Minimize costly CPU-GPU transfers and exploit PRM's massive parallelism.



% \end{itemize}
\section{Problem Statement}
\label{sec:problem_statement}
We model a robot by its configuration $\bfq \in \calC = \cfree \cup \cspace_{\rm occupied} \subseteq \mathbb{R}^k$,  
where $k$ is the number of controllable degrees of freedom including both the base and joints. The set $\cfree$ denotes the collision-free subset of the configuration space, while $\cspace_{\rm occupied}$ corresponds to states in collision. The robot operates in a physical workspace $\workspace \subseteq \mathbb{R}^3$, and motion planning is the problem of computing feasible trajectories in this high-dimensional space.  

The classical objective is to compute a continuous path $\pi : [0,1] \rightarrow \cfree$ that connects a start configuration $\pi(0) = \cstart$ to a goal region $\pi(1) \in \cgoal$.  
In context-aware motion planning, simply avoiding collisions is insufficient; robots are also required to follow additional task-oriented requirements. Such requirements demand that the planner possesses a high-level understanding of its planning environment, which greatly increases the complexity of planning.  

To capture this, we assign to each path $\pi \in \Pi$ a composite cost functional  
\begin{equation}
c(\pi) = c_m(\pi) + \sum_{j \in J} \alpha_j \, c_j(\pi),
\label{eq:general_cost}
\end{equation}
where $c_m(\pi)$ is the \emph{motion cost}, typically the path length in configuration space,
\begin{equation}
c_m(\pi) = \int_0^1 \lVert \dot{\pi}(t) \rVert \, \mathrm{d}t,
\label{eq:motion_cost}
\end{equation}
and $J$ indexes the set of auxiliary task costs. Each term $c_j(\pi)$ quantifies a task-specific requirement, while $\alpha_j \in \mathbb{R}$ is its relative weight. These auxiliary terms, $\alpha_j \, c_j(\pi)$ where $ j\in J$, can depend on robot state, environmental geometry, or external task variables, and are often nonlinear, high-dimensional, and computationally expensive. This poses a significant challenge: evaluating $c_j(\pi)$ may involve repeated calls to expensive models (e.g., neural networks or simulators) that would compound across all states in the PRM.  

In this work, we address these challenges by introducing a GPU-parallelized pipeline that enables efficient PRM-based planning under such cost-augmented formulations.  



\section{Background and Related Work}
\label{sec:related_work}

A key challenge in motion planning for autonomous robots is efficiency. High-speed planning enables continuous replanning, allowing robots to adapt to unstructured environments. Designing more efficient algorithms and leveraging hardware acceleration are two orthogonal but complementary directions for achieving real-time motion planning. While there has been significant progress in developing faster algorithms~\cite{kuffner2000rrt, sucan2011sampling, otte2016rrtx, gammell2020batch}, our work focuses on hardware acceleration and leverages GPUs to exploit parallelism in classical motion planners such as PRM~\cite{kavraki2002probabilistic}.

Motion planners should also maintain awareness about the context in which they are planning. Rather than merely planning a trajectory, the planner has to consider contextual information. For instance, designing planners that are human-aware or perception aware is critical for enabling autonomous robots to operate effectively in dynamic and unstructured settings.

\subsection{Parallelism and Hardware Acceleration}\label{sec:hardware}

Early efforts on parallel SBMP algorithms were initially discussed in~\cite{henrich1997fast}. Parallelism in SBMP can be applied at different levels and for different purposes. For example, Parallel PRM~\cite{amato1999probabilistic} exploited parallelism for vertex generation and k-nearest-neighbor connections.~\cite{wedge2008heavy} improved average-case performance by running multiple planners in parallel, while~\cite{plaku2005sampling} constructed a forest of tree-based planners in parallel. More recently, VAMP~\cite{thomason2024motions} parallelized collision checking and incorporated early termination strategies to dramatically speed up planning. A parallel RRT* was also proposed in~\cite{xiao2017parallel}. Those works~\cite{amato1999probabilistic, wedge2008heavy, plaku2005sampling} are platform-independent, either due to their generalizability or because GPUs were not yet widely available when the algorithms were introduced.

% Beyond SBMP, parallelism has been widely applied in model predictive control~\cite{bhardwaj2022storm, hyatt2020parameterized} and trajectory optimization~\cite{sundaralingam2023curobo}.

CPUs and GPUs are the two primary hardware platforms for parallelism. CPUs typically rely on single instruction, multiple data
(SIMD) parallelism to accelerate vectorized operations, while GPUs employ single instruction, multiple threads (SIMT) to execute massive numbers of lightweight threads in parallel. CAPT~\cite{ramsey2024collision} is a representative work that leverages SIMD primarily to avoid costly data transfers between CPUs and GPUs. This design choice enables planning for high-DoF robots in highly cluttered and dynamic environments. Subsequent works~\cite{wilson2025nearest, wilson2025aorrtc} adopt SIMD for similar reasons. In contrast, many other approaches~\cite{huang2025prrtc, hu2025cprrtc, sundaralingam2023curobo} utilize SIMT parallelism, taking advantage of the vast number of GPU threads to efficiently compute projections, signed distance fields, and related operations.

The GPU is a natural choice for our work because of PRM's inherent parallelism and our use of neural cost functions. Apart from the initial transfer of sensor data to the GPU and the final transfer of the roadmap back to the CPU, all computations are performed on the GPU, keeping data transfer costs minimal.

\subsection{Context-Aware Motion Planning}
Beyond basic collision avoidance, motion planners often need to account for the specific context and task in which they are deployed. This context-aware motion planning is crucial, as different contexts impose different requirements and constraints, especially in HRI or HRC. In such settings, the robot must not only plan smooth collision-free paths but also take into account human intentions, safety, privacy, and many other factors. For example, in human-aware motion planning~\cite{mainprice2011planning, rajendran2021human}, the most common constraints are visibility and distance. Visibility constraints ensure that the robot remains within the human’s field of view for as long as possible to avoid surprise, while distance constraints maintain a safe separation between the robot and the human. In privacy-aware motion planning~\cite{luo2020privacy, shome2023robots, martin2016privacy}, an additional constraint is introduced: the robot must avoid entering certain areas due to privacy restrictions, or ensure that its camera does not capture the human’s face. In intention-aware motion planning~\cite{park2019planner, ren2022human, liu2024intention, chen2025int2planner}, a common constraint is to incorporate predictions of human intention (often obtained from neural networks~\cite{park2019planner, liu2024intention}) to plan trajectories that either avoid the human or enable effective collaboration.

Another important class of context-aware motion planning is perception-aware motion planning. In perception-aware motion planning, the main objective is for the robot to plan a path such that the viewpoints from its onboard camera are optimized for a given perception task such as 3D scene reconstruction~\cite{zhao2022perception}, object tracking~\cite{falanga2018pampc, tordesillas2023deep}, object pose estimation~\cite{hu2022view}, or object monitoring~\cite{meng2025look}. To achieve this, perception-aware planners typically define a cost function—often derived from neural networks—associated with the robot or camera pose; the planner jointly optimizes both motion cost and perception cost. For instance,~\cite{meng2025look} incorporates estimates of object monitoring performance from a neural surrogate model into the planning process, while~\cite{hu2022view} leverages 6D object poses predicted by a pose estimation network to guide planning.

For most of the context-aware motion planners discussed above, the primary constraint or objective for a given context can be formulated as a cost function defined over the robot configurations and other context-specific variables—for example, the object pose in perception-aware motion planning~\cite{meng2025look,falanga2018pampc} or human intention in human-aware motion planning~\cite{park2019planner}. Moreover, many of these cost functions can be represented by, or directly implemented as, neural networks~\cite{meng2025look, park2019planner, hu2022view, zhao2022perception, chen2025int2planner, martin2016privacy}, which naturally benefit from GPU-based hardware acceleration. This observation demonstrates the need for a GPU-accelerated variant of PRM that achieves real-time performance and generates paths optimized under a neural cost function. 

% Beyond high-performance, real-time planners must also accommodate a wide range of constraints, since field environments are frequently messy and unpredictable. Accordingly, there is a large body of real-time planning research that addresses requirements beyond basic collision avoidance. For example, ~\cite{liu2023task} incorporate the uncertainty of human motion into the planning process, while ~\cite{sisbot2012human} account for human kinematics, preferences, and field-of-view (FOV) to enable safe human-robot collaboration (HRC). ~\cite{xi2024lightweight} develop a highly-efficient reinforcement learning algorithm designed to satisfy strict power consumption constraints for unmanned aerial vehicles (UAVs). ~\cite{papachristos2019localization} and ~\cite{bartolomei2020perception} use trajectory optimization and segmentation maps, respectively, to reduce localization uncertainty for UAVs. CuRobo ~\cite{sundaralingam2023curobo}, an optimization-based planner that allows for constrained planning with locked joints, and cpRRTC ~\cite{hu2025cprrtc}, a derivative of RRT-Connect ~\cite{kuffner2000rrt} that enables end-effector constrained planning, are two state-of-the-art constrained planners that are parallelized for use on GPUs. While these frameworks are capable of replanning on-the-fly while also incorporating specialized constraints, each is tailored specifically for its target application and is not designed to accommodate a broad set of user-defined cost functions.


% This motivates the need to combine real-time general-purpose planning frameworks with the ability to handle a wide array of constraints. Sampling-based planners, such as the probabilistic roadmap (PRM) and rapidly-exploring random tree (RRT) connect, introduced by ~\cite{kavraki2002probabilistic} and ~\cite{kuffner2000rrt}, respectively, can be extensively parallelized ~\cite{amato1999probabilistic,huang2025prrtc} and do not require complex training procedures. At the same time, learning-based planning methods have the potential to quantify abstract environmental qualities that can be used to satisfy novel or non-traditional constraints ~\cite{you2025human, saha2024edmp, sisbot2012human}. Since PRM and related sampling-based planners are already well-suited for handling traditional planning restrictions (e.g., collision avoidance or joint limits), a real-time planning framework that fuses the strengths of learned and traditional objectives, while also supporting user-defined custom cost functions, is highly desirable. 

% \begin{enumerate}
%     \item Perception-aware planning: We sometimes want the robot not only to move to a certain configuration but also to continuously monitor an object from a specific angle while moving. The perception score, which measures the robot's ability to detect objects or humans along a path, can be approximated by a neural network. 
%     \item Human distance aware planning~\cite{mainprice2011planning, rajendran2021human}: In real-world scenarios, a robot must maintain a certain distance from humans. This is crucial for safety in human-robot collaboration. The framework can incorporate this by using a cost function that penalizes configurations where the robot is too close to a human.
%     \item Privacy-aware planning: When humans are present, we might want the robot to avoid looking at their faces to protect their privacy. Such a constraint can be quantified by a neural network and integrated into the planning process.
%     \item Integrating with LLMs or VLMs: Humans can describe their requirements or constraints using natural language. These descriptions can be interpreted by Large Language Models (LLMs) or Vision-Language Models (VLMs), which could then be used to rate the cost of each node, allowing the planner to adapt to a wide range of user-defined constraints.
% \end{enumerate}


% Historically, researchers have developed a number of methods for achieving real-time planning frequencies, such as anytime planning strategies ~\cite{vannoy2008real, van2006anytime}, closed-loop implementations of kinodynamic planners ~\cite{frazzoli2002real, kuwata2009real}, and decomposition of the planning process into manageable subproblems ~\cite{brock2001decomposition}. In this work, we focus on hardware acceleration, a method of reducing planning times that has been employed since the early 1990s and the era of grid-based planners ~\cite{lengyel1990real, lozano1991parallel}.

% There are several types of hardware acceleration techniques. Multithreading on multi-core central, graphics, and tensor processing units (CPU, GPU, and TPUs) enables the execution of single-instruction multiple-thread (SIMT) program architectures, which run one instruction (e.g., “sample a random configuration”) across many concurrent threads. Single-instruction multiple-data (SIMD) techniques allow programmers to apply the same instruction to large arrays of data simultaneously within a single thread. 
% Field-programmable gate arrays (FPGAs) and the creation of custom micro-architectures are two additional hardware techniques commonly used to improve program execution.


% These approaches- SIMD, SIMT, FPGA, and custom micro-architectures- havqe all been successfully used to accelerate motion planning in ~\cite{thomason2024motions, ramsey2024collision, wilson2025nearest}, ~\cite{sundaralingam2023curobo, lee2024gpu, park2013real, pan2012gpu}, ~\cite{atay2006motion, murray2016microarchitecture}, and ~\cite{huang2022hardware}, respectively, often by one or more orders of magnitude. Hardware acceleration is a broadly applicable method of improving planning efficiency; it has been applied to grid-based planning ~\cite{lengyel1990real, lozano1991parallel}, sampling-based planning ~\cite{thomason2024motions, huang2025prrtc}, and optimization-based planning ~\cite{sundaralingam2023curobo, lee2024gpu, park2013real}. Learning-based planners, similarly, are typically reliant on GPU/TPU hardware during training and, to a lesser extent, during deployment ~\cite{ichnowski2020deep}.

\section{\ourplanner Algorithm}\label{sec:method_baseline}

In this section, we detail the architecture and implementation of \ourplanner. We make two major modifications to the standard implementation of PRM. First, in Sec.~\ref{sec:construction}, we parallelize roadmap construction on GPU to generate motion plans in milliseconds. Second, in section~\ref{sec:ncf}, we use a neural cost function to assign context-specific costs to each node. These costs are used during the PRM query phase to generate paths that are context-aware. A high-level view of this process is illustrated in Fig.~\ref{fig:prm_flowchart}, which includes the transfer of the PRM from GPU to CPU. On CPU, we use an implementation of a graph search algorithm, such as $\mathcal{A}$*~\cite{hart1968formal} or Dijkstra's~\cite{dijkstra2022note}, to get a solution from the roadmap.


% Second, we perform batched forward kinematics on GPU to compute the camera joint positions that steer the camera towards the object of interest for each node in the roadmap, and update the nodes accordingly. 

\begin{algorithm}[!ht]
\caption{Parallel Roadmap Construction}
\label{alg:roadmap}
\begin{algorithmic}[1]
\Procedure{PRM-Construction}{$N, T_{\rm term}, T_{\rm sample}, K$}
\State { /* $N$ denotes the number of threads. */}
\State { /* $T_{\rm term}$ denotes the terminate condition. */}
\State { /* $T_{\rm sample}$ denotes the max attempts to sample a configuration. */}
\State { /* $K$ denotes the number of neighbors for k-NN. */}
% \State {Initialize an empty graph $G = (V, E)$}
\State {$V, E\leftarrow \emptyset$} 
\While{not $T_{\rm term}$}
    \State { /* Step 1: Sample new vertices in parallel */}
        % \State{$C_{\rm rand} \leftarrow \emptyset$} \Comment{Shape: $N\times {\rm dof}$}
    \State{$C_{\rm rand} \leftarrow \Call{sample-parallel}{N, T_{\rm sample}}$\label{line:sample-parallel}}
    \State{$V\leftarrow V\cup\{c_{\rm rand}\}$}
    \State{}
    \State { /* Step 2: Compute NN in parallel */}
    \State{$C_{\rm near} \leftarrow \Call{KNN-search}{C_{\rm rand}, V, N, K}$\label{line:knn}}
    \State{}
    \State { /* Step 3: Connect edges in parallel*/}
    \State{$E_{\rm new} \leftarrow \Call{connect-edges}{C_{\rm rand}, C_{\rm near}, N}$\label{line:connect}}
    \State{$E\leftarrow E\cup E_{\rm new}$}
    \EndWhile
\EndProcedure
\end{algorithmic}
\end{algorithm}


\subsection{Parallelized Roadmap Construction}\label{sec:construction}
To facilitate discussion, we first give a brief introduction to the parallelized PRM architecture; each stage of roadmap construction is naturally parallelizable on GPUs. We develop custom CUDA kernels to accelerate sampling, neighbor search, and collision checking processes with SIMT parallelism and expose these kernels to the rest of our implementation via Python bindings. We provide pseudocode in Algorithm~\ref{alg:roadmap}.

For a roadmap with $N$ nodes, we launch a state generation kernel (see line~\ref{line:sample-parallel}) with $N$ threads. Each thread is responsible for sampling a new configuration $c_{\rm rand}$. Since $c_{\rm rand}$ may be invalid, we set a maximum number of attempts $T_{\rm term}$ and maintain a counter $t_{\rm atp}$ to track retries. If no valid configuration is found within the limit, $c_{\rm rand}$ is flagged as invalid and is omitted from the roadmap. K-nearest-neighbor (k-NN) computation, line~\ref{line:knn}, follows a similar strategy: the k-NN kernel is launched with $N$ threads and each thread is responsible for finding the $K$ nearest neighbors of one configuration in the PRM. We fix $K$ rather than a search radius $R$ in an effort to prevent thread divergence and improve connectivity in situations where a configuration may have few neighbors within $R$. 

We use separate kernels for edge interpolation and collision checking; line~\ref{line:connect} represents these complementary kernels as one unit. We employ a lazy strategy in which the edges between a node and each of its $K$ neighbors are locally connected and assumed to be valid. Then, the collision checker removes any invalid edges; it also transforms meshes into geometric primitives to perform validity checks. The edge-connection kernel is launched with $N \times K$ threads, where each thread connects one node to one of its $K$ neighbors. The collision checking kernel is similarly launched with $N \times K$ threads; each thread checks for collision between one edge and every obstacle in the environment.

\begin{figure}[!ht]
    \centering
    \includegraphics[width=1.0\columnwidth]{image/prm_flowchart.pdf}
    \caption{
    This figure illustrates the pipeline of our planner. Each kernel (shown in the green dotted boxes) corresponds to a component of PRM construction: sampling, $k$-nearest neighbor search, edge connection, and scoring. The data is then transferred to the CPU, where our planner uses graph search to compute a path that balances motion cost with neural cost.
    }
    \label{fig:prm_flowchart}
\end{figure}

\subsection{Neural Cost Function}\label{sec:ncf}

We assign a cost to each configuration based on a pretrained neural network, $\mathcal{N}$, immediately after PRM construction. The network takes a robot configuration $\bfq$ and a latent variable $z$ as input and outputs a cost $c \in [0,1]$, where lower values indicate more preferred configurations. This latent variable $z$ can capture any other contextually-relevant information about the environment; for example, in human-aware planning $z$ would represent the pose of a human collaborator. The objective of a neural network can be flexible (e.g., safety, privacy, human intention, etc.). What makes this approach more powerful is that, for a given configuration $\bfq$, the final cost $c(\bfq,z)$ can be expressed as the sum of multiple networks.

\begin{equation}
    c(\bfq,z) = \sum_{j\in J}\gamma_j\mathcal{N}_j(\bfq, z)
\end{equation}

As most graph search methods operate on edge costs rather than node costs, we define a general transformation $\mathcal{T}$ that maps the costs of the two endpoint configuration of an edge, together with any intermediate configurations or latent variables, into an edge cost: 
\begin{equation}
    C_{\text{neural}}(e) \;=\; \mathcal{T}\Big(c(\bfq_i, z_i),\, c(\bfq_j, z_j),\, \{c(\bfq_k, z_k)\}_{k \in \mathcal{I}(e)}\Big),
\end{equation}
where $e=(\bfq_i,\bfq_j)$ denotes an edge in the roadmap, $c(\bfq,z)$ is the node cost as defined in~(1), and $\mathcal{I}(e)$ is an optional set of intermediate configurations sampled along $e$. The function $\mathcal{T}$ can be chosen flexibly (e.g., average, maximum, or learned aggregation), enabling the integration of node-level neural costs into edge-level graph search.


This new cost for each edge $C_{\text{neural}}(e)$ is combined with the roadmap graph, allowing for graph search algorithms to optimize for it and any other relevant weight (e.g. motion cost). To incorporate this into planning without restricting to a linear blend, we define the final edge cost as an aggregation of the original motion cost and the neural cost:
\begin{equation}
    C_{\mathrm{total}}(e; z)
    \;=\;
    \Phi\!\big(C_{\mathrm{motion}}(e),\, C_{\mathrm{neural}}(e; z);\, \eta\big),
\end{equation}
where $\Phi:\mathbb{R}_{\ge 0}^2\times\Xi\!\to\!\mathbb{R}_{\ge 0}$ is any monotone aggregator (e.g., sum, max, $p$-norm, log-sum-exp, product, lexicographic smoothing, or a learned MLP), and $\eta$ collects its hyperparameters. This combined cost allows graph search algorithms to optimize simultaneously for motion efficiency and neural costs such as safety, visibility, or human-awareness. This method is therefore a general framework to incorporate different context-specific constraints. Furthermore, our ability to replan in real-time theoretically allows each cost weight, $\gamma_j$, to be updated dynamically. Dynamic weight updates would allow the properties of our motion plans to be changed on the fly. For example, if a robot is executing tasks in an empty room when a human enters, the planner could receive a signal from sensors to increase the weight $\gamma_j$ on the privacy cost $c_j$ from Eq.~\ref{eq:general_cost}. This would put the robot in ``privacy-aware'' planning mode and modify all of the robots behavior until the human was no longer present, at which point $\gamma_j$ would return to its unmodified value.



% --------- OLD DESCRIPTION OF CUDA -----------
% The sampling and collision-checking steps are naturally parallelizable on GPUs. We develop custom CUDA kernels to accelerate the roadmap construction process. The entire construction is executed on the GPU, with the results exposed to the rest of our PRM-NC implementation through Python bindings.

% Specifically, we use SIMT (Single Instruction, Multiple Threads) parallelism to run $N$ threads in parallel. Each kernel is launched with a single thread responsible for sampling a new configuration $c_{\rm rand}$. Since a sampled configuration may be in collision with obstacles, we set a maximum number of attempts $T_{\rm term}$ and maintain a counter $t_{\rm atp}$ to track retries. Valid samples from all threads are stored in a pre-allocated memory buffer $C_{\rm rand}$. If no valid configuration is found within the limit, the corresponding entry is flagged as invalid. After sampling, the new configurations in $C_{\rm rand}$ are added to the PRM as vertices.

% the Our parallel neighbor search is a standard brute-force implementation of k-nearest neighbors; it uses the \( \ell_2 \) distance computed in the subset of $\cspace$ that does not include the robot’s camera joints\Weihang{Mention later?} \Emiliano{not necessarily only camera joints, but the subset of joints to use in constraint projection}.  Our parallel roadmap construction could be made more elegant; however, we leave this for a future work since our goal with this parallelization is simply to enable real-time planning frequencies.

% The algorithm maintains a graph $G = (V, E)$. To expand $G$, it first samples a random configuration $c_{\rm rand}$ in the configuration space, which becomes a candidate vertex to add to the graph. Next, a $k$-nearest neighbor search is performed to find the set of nearest neighbors of $c_{\rm rand}$, denoted by $C_{\rm near}$. The algorithm then attempts to connect $c_{\rm rand}$ to each element in $C_{\rm near}$ through collision checking. If a collision-free path exists between two vertices, the corresponding edge is added to $G$.

% --------- OLD SPLIT PSEUDOCODE ---------
% \subsection{Parallel Roadmap Construction}\label{sec:construction}
% \begin{algorithm}\label{alg:sample}
%     \caption{Sampling}
%     \begin{algorithmic}[1]
%         \Procedure{sample-parallel}{$N, T_{\rm sample}$}
%         \State { /* $N$ denotes the number of threads. */}
%         \State { /* $T_{\rm term}$ denotes the terminate condition. */}
%         \State{$C_{\rm rand} \leftarrow \emptyset$}
%         \For{$n=0\rightarrow (N-1)$} \Comment{In parallel}
%             \State{$k_{\rm atp} \leftarrow 0$}
%             \While{$t_{\rm atp} < T_{\rm sample}$}  
%                 \State{$c_{\rm rand}\leftarrow \Call{sample-config}{\null}$}
%                 \If{\Call{is-valid}{$c_{\rm rand}$}}
%                     \State{$C_{\rm rand}[n] \leftarrow c_{\rm rand}$}
%                     \State{\texttt{break}}
%                 \Else
%                     \State{$t_{\rm atp} \leftarrow t_{\rm atp} + 1$}
%                 \EndIf
%             \EndWhile
%         \EndFor
%         \State{\Call{sync}{\null}}
%         \State{\Return $C_{\rm rand}$}
%         \EndProcedure
%     \end{algorithmic}
% \end{algorithm}

% \begin{algorithm}\label{alg:knn}
%     \caption{KNN}
%     \begin{algorithmic}[1]
%         \Procedure{KNN-parallel}{$C_{\rm rand}, V, N, K$}
%         \State { /* $C_{\rm rand}$ denotes the new vertices. */}
%         \State { /* $V$ denotes the vertices in the PRM. */}
%         \State { /* $N$ denotes the number of threads. */}
%         \State { /* $K$ denotes the number of neighbors for k-NN. */}
%         \State{$C_{\rm near} \leftarrow \emptyset$} \Comment{Shape: $K\times N\times {\rm dof}$}
%         \For{$n=0\rightarrow (N-1)$} \Comment{In parallel}
%             \State{$C_{near}[n]\leftarrow\Call{KNN}{V, C_{rand}[n], K}$}
%         \EndFor
%         \State{\Call{sync}{\null}}
%         \State{\Return $C_{\rm near}$}
%         \EndProcedure
%     \end{algorithmic}
% \end{algorithm}

% \begin{algorithm}\label{alg:knn}
%     \caption{Connect edges in parallel}
%     \begin{algorithmic}[1]
%         \Procedure{connect-edges}{$C_{\rm rand}, C_{\rm near}, N$}
%         \State { /* $C_{\rm rand}$ denotes the new vertices. */}
%         \State { /* $C_{\rm near}$ denotes the k nearest vertices of $C_{\rm rand}$. */}
%         \State { /* $N$ denotes the number of threads. */}

%         \State{$E_{\rm new}\leftarrow \emptyset$}
%         \For{$n=0\rightarrow (N-1)$} \Comment{In parallel}
%         \State{$c_1 \leftarrow C_{rand}[n]$}
%         \For{$m=0\rightarrow (K-1)$}
%             \State{$c_2 \leftarrow C_{near}[n][m]$}
%             \If{\Call{is-valid-path($c_1, c_2$)} }
%                 \State{$E_{\rm new}\leftarrow E_{\rm new}\cup \{(c_1, c_2)\}$}
%             \EndIf
%         \EndFor
%     \EndFor
%     \State{\Return $E_{\rm new}$}
%         \EndProcedure
%     \end{algorithmic}
% \end{algorithm}

% --------- OLD MONOLITHIC PSEUDOCODE ---------
% \begin{itemize}
% \item  We use a lightweight multi-layer perceptron (MLP) as our neural representation of the abstract cost functions.
% \item We distill knowledge from a large perception model (e.g. YOLO-?) into our MLP. The MLP is trained on the perception scores provided by the larger model.
% \item The MLP takes the difference in pose between the robot's camera and the object to-be-observed as input. This difference in pose is represented as delta(x,y,z,qx,qy,qz,qw). The model outputs a perception score between 0 and 1.
% \end{itemize}


% \SetKw{KwFunc}{Functions:}
% \KwData{$N$ (number of nodes), $DIM$ (node dimension), $K$ (neighbors), $INTERP$ (interpolation steps), $RS$ (max resampling attempts)}
% \KwFunc{$\Phi$ (node constrainer), $\Psi$ (node scorer)}\nonumber
% \BlankLine
% % Declare arrays
% Nodes[$N$, $DIM$], Neighbors[$N$, $K$], Edges[$N$, $K$, $DIM$, $INTERP\_STEPS$], Validity[$N$, $K$], Scores[$N$] $\to$ GPU\
% \BlankLine
% \textbf{Launch SAMPLER with $N$ threads:}
% \For{$i \gets 1$ \textbf{to} $N$ \textbf{in parallel}}{
%     Node $\gets$ RANDOM\_CONFIG()\;
%     \While{State invalid \textbf{and} $attempts$ $<$ $RS$}{
%         Node $\gets$ RANDOM\_CONFIG()\;
%     }
%     Nodes[$i$] $\gets$ Node\;
% }
% \BlankLine
% \BlankLine
% \textbf{Launch kNN, $N$ threads:}
% \For{$i \gets 1$ \textbf{to} $N$ \textbf{in parallel}}{
%     Neighbors[$i$] $\gets$ NEAREST\_NEIGHBOR(States[$i$], $K$)\;
% }
% \BlankLine
% \BlankLine
% \textbf{Launch EDGE CONSTRUCTION, $kN$ threads:}
% \For{$i \gets 1$ \textbf{to} $N$, $j \gets 1$ \textbf{to} $K$ \textbf{in parallel}}{
%     Edge $\gets$ CONNECT(Nodes[$i$], Neighbors[$i,j$])\;
%     Edges[$i,j$] $\gets$ Edge\;
% }
% \BlankLine
% \BlankLine
% \textbf{Launch IS VALID, $kN$ threads:}
% \For{$i \gets 1$ \textbf{to} $N$, $j \gets 1$ \textbf{to} $K$ \textbf{in parallel}}{
%     Validity[$i,j$] $\gets$ isValid(Edges[$i,j$])\;
% }
% \BlankLine
% \BlankLine

% Nodes $\gets \Phi(\text{Nodes})$ \\
% Scores $\gets \Psi(\text{Nodes})$ \\
% (States, Neighbors, Edges, Validity, Scores) $\to$ CPU\\\
% PRM $\gets$ ASSEMBLE(States, Neighbors, Edges, Validity, Scores)\\\
% \Return PATH($q_{\text{start}}$, $q_{\text{goal}}$, PRM)

\section{Experiments}

We evaluate our GPU-accelerated planner using a perception-aware cost function as a representative example. 
In this case, the auxiliary term $c_p(\pi)$ measures the robot’s ability to detect objects or humans along a path. 
Such perception quality depends on factors like viewing angle and distance, which are difficult to model analytically. 
Instead, we approximate the score with a neural network that can be efficiently parallelized on the GPU. 
Formally, for each configuration $\bfq \in \cfree$ and object $o \in \calO$, we define
\begin{equation}
p(\bfq,o) \in \mathbb{R}_{\geq 0},
\end{equation}
as the predicted perception score, and set
\begin{equation}
c_p(\pi) = \int_0^1 \sum_{o \in \calO} w_o \, p(\pi(t),o)\,\mathrm{d}t.
\end{equation}

We demonstrate our planner in both simulation and real-robot experiments with the Hello Robot Stretch~2. Our implementation uses Isaac Sim~\cite{nvidia2022isaacsim} for simulation and visualization, and all experiments are run on an Intel i7-12700K CPU with an NVIDIA RTX 4090 GPU. For simulation, the camera is configured with a resolution of $640\times 480$, a field of view of $42.5^\circ$, and a clipping range of $[0.3,6]$ meters. We use $\mathcal{A}$* search on the CPU to find solutions in our PRM.

The neural cost function $p(\bfq,o)$ in our experiments is modeled as a multilayer perceptron (MLP)~\cite{rumelhart1986learning}. The input features include the relative camera–object pose and an object label. This framework and neural cost function can be generalized to a wide range of applications; here, we present one implementation to illustrate our pipeline. For example, in human-aware or intention-aware motion planning, the neural cost function could instead be a network trained for human activity prediction and recognition~\cite{kumar2024human}. Moreover, thanks to the scalability of our approach, the architecture of the neural network is flexible and could vary in depth, width, or design depending on the target application. Furthermore, our framework can accommodate any GPU-based cost function, such as ray tracing~\cite{parker2013gpu}, signal processing~\cite{mccool2007signal}, or large matrix operations~\cite{ferreira2011bayesian}, rather than only neural networks. 

For this perception task, \ourplanner executes roadmap building for Stretch's non-holonomic ${SE(2)}$ state space. Once the roadmap is built, we steer Stretch 2's camera to point at an object for every state in the roadmap. This ensures that the perception score assigned to each node by the neural cost function is valid. Maximizing the perception function costs across the roadmap ensures that the $c_p$ assigned to every path $\pi$ in $\Pi$ is optimized.

% point the camera at the object of interest by projecting 


% a decoupled set of two joints is used to project the camera against an object of interest. 

% Once the roadmap, $\mathbf{G}$, is built, we must ensure that the robot’s camera is properly oriented to look at the object of interest for every node in $\mathbf{G}$. This ensures that the perception score assigned to each node by the neural cost function is optimized.  Maximizing the perception scores uniformly across $\mathbf{G}$ ensures that the $c_p$ assigned to every path $\pi$ in $\Pi$ is both accurate and unbiased.  In other words, since the perception scores are embedded in the heuristic used during graph search, their values must be consistent across $\mathbf{G}$; otherwise, the paths output by the planner would not optimally satisfy the neural cost function.

Using the readily available PRM information on GPU memory, we include a parallelized forward kinematics (FK) function on GPU that accelerates the projection of nodes in the roadmap to satisfy the camera view on the object. We use the Pytorch-FK library ~\cite{Zhong_PyTorch_Kinematics_2024} to allow both GPU based FK and transfer of robot data in PyTorch tensors. This GPU-FK functionality integrates with our GPU roadmap construction and allows the planner to accommodate a wide range of additional robot constraints for future experiments as well.

% The notion of an EE is arbitrary: it could be a gripper, camera, or surgical instrument, and may come with different constraints on its position or orientation. For example, it may be beneficial to orient a camera to look at an object, while a scalpel would need to stay perpendicular to a cut surface. It is important to note that this planning component is optional and only needs to be used if the user desires to constrain the EE. We discuss our specific use of the parallel FK functions in ~\ref{sec:expPFK}, where we describe our experimentation with perception-constrained planning scenarios.


\subsection{Simulation Experiments}

We first evaluate our planner in a simulated office environment with three different objects-of-interest (OoI): a cup, monitor, and human. We randomly sample 25 start and goal pairs from regions on opposite ends of the environment and perform 100 trials per pair per OoI. We compare our GPU-accelerated planner against a CPU equivalent planner based on~\cite{meng2025look}. This baseline implementation uses the same GPU based FK and inference on its generated PRM, but executes state sampling and roadmap building on CPU using OMPL’s PRM implementation~\cite{sucan2012open}. The baseline uses the same perception-aware cost function and search parameters, and a CPU-equivalent collision checking method. 

We also evaluate the scalability of our planning by comparing planning times for PRMs with different numbers of nodes. We do so in the same office environment with the same start and goal pairs; Fig. \ref{fig:perf_scaling} presents the resulting timing statistics. As shown in Fig.~\ref{fig:perf_scaling}, the build and query time of the PRM planner increases consistently with the number of nodes in the roadmap. At small roadmap sizes (up to about $5{,}000$ nodes), the time grows from $10^{-4}$ to just under $10^{-2}$ seconds, reflecting the cost of adding connectivity to an initially sparse graph. In the intermediate range (around $10{,}000$ nodes), the time reaches $10^{-1}$ seconds, with a slower rate of increase as edge construction dominates. Beyond $10{,}000$ nodes, however, the cost rises more steeply, approaching $1$ second at $20{,}000$ nodes, indicating the quadratic growth of potential edges and the higher complexity of managing large graphs. Overall, the results highlight a clear trade-off between roadmap density, which improves connectivity and path quality, and build time, which scales from sub-linear at smaller sizes to near super-linear at larger scales.


\begin{figure}[!ht]
    \includegraphics[width=0.4875\textwidth]{image/dual_axis_performance_boxplot__.pdf}
    \caption{
        This figure shows the performance of our planner as the number of nodes in increased - the y-axis is shown in \textbf{logarithmic} scale. The statistics represent the combined roadmap construction and trajectory generation time; they were gathered for Hello Robot's Stretch 2 in our simulated office environment. The planner can sustain re-planning frequencies of \>10 hertz for roadmaps with as many as 5000 nodes.
    }
    \label{fig:perf_scaling}
\end{figure}


\begin{table}[t]
\centering
\resizebox{\columnwidth}{!}{
\begin{tabular}{lccc|c}
\toprule
 & \multicolumn{3}{c}{Motion Planning} & \multicolumn{1}{c}{Perception} \\ 
\cmidrule(lr){2-4} \cmidrule(lr){5-5} 
 \multirow{2}{*}{Method} & Build Time & Plan. Time & Path Len. & Det. Rate \\
 & (s) & (s) & (rad) & (\%) \\ 
\midrule
\rowcolor{gray!1} CPU-Baseline & 2.62 & 0.11 & 15.65 & 0.637 \\ 
\rowcolor{gray!10} \ourplanner & \textbf{0.0166} & \textbf{0.0042} & \textbf{13.61} & \textbf{0.697} \\ 
\bottomrule
\end{tabular}}


\caption{Planning results for Stretch on a simulation environment. The results are averaged over 100 different motion planning problems and a PRM with 1500 nodes. ``Build Time'' denotes the time to create the roadmap, ``Plan. Time'' the average planning time, ``Path Len.'' the average path length and ``Det. Rate'' the percentage of frames with the object of interest detected.}
\label{tab:planning-results}
\end{table}

Table \ref{tab:planning-results} shows the planning and perception statistics for \ourplanner against the CPU baseline. Results for the perception comparison were obtained by benchmarking object detection using YOLOE~\cite{wang2025yoloe} prompted with a brief description of the objects of interest. The results highlight the significant speedup of over 130 times on PRM build and planning time as the most important improved metric. While build time is accelerated by PRM parallelization on node and edge validity checking, planning time is also improved by faster edge checks on start and goal connections to the roadmap. Path quality is preserved with similar average path lengths and performance on the perception task.

In our experiments, we do not include direct comparisons with prior GPU-parallelized PRM or RRT methods because these planners were not designed to incorporate neural costs, making such comparisons unfair and uninformative. For instance, cpRRTC~\cite{hu2025cprrtc} uses a GPU-accelerated tree-based structure which is well suited for fast exploration; however, unlike a roadmap-based planner it is fundamentally difficult to optimize with respect to global cost functions. Similarly, prior works on GPU-accelerated PRM~\cite{blankenburg2020towards,pan2010g,pan2010efficient,pan2012gpu,amato1999probabilistic} focus exclusively on accelerating classical PRM operations. These systems lack the necessary mechanisms, such as parallel forward kinematics and cost evaluation pipelines, for integrating neural costs into the planning process. As a result, a direct comparison would primarily measure low-level GPU performance rather than addressing the central contribution of our work: a GPU-native planner that seamlessly incorporates neural cost functions for context-aware motion planning.


\subsection{Real Robot Experiments}

We further evaluate the effectiveness of our planner in a real-robot environment. The robot is placed in an indoor environment with a sofa, a human sitting on a black chair, and a white chair; both chairs and the human are free to move. The robot must move to the opposite side of the room while continuously monitoring the human's face and avoiding dynamic obstacles. The robot must prioritize paths that allow it to see the human's face, rather than the human in general, as long as they are valid. Therefore, it must replan whenever the human or other obstacles change either their position or orientation. We use the motion capture system Vicon Tracker~\cite{pfister2014comparative} to send the poses of both the objects and the human to the planner in real time. The planner can rebuild the roadmap and deliver a new trajectory to the robot at up to 75~Hz, but replanning is triggered only when the human or chairs undergo a translation or rotation that exceeds a predefined threshold.

As illustrated in Fig.~\ref{fig:real_robot_dynamic}, the robot continuously adapts its path in response to changes in the human’s pose and in the environment. When the human initially faces right, the robot selects a trajectory along the right side (red path). As the human turns to face left, the planner dynamically updates the path to the opposite side (orange path). Later, when the human moves the white chair, the robot replans once again in real time to avoid the newly introduced obstacle (green path). This experiment highlights the ability of our planner to react swiftly to both human motions and environmental disturbance while maintaining perception-awareness.

An additional experiment was designed to test the capability of the planner to react to aggressive changes in the environment. As shown in Fig.~\ref{fig:real_robot_thrown} the robot is tasked with visually tracking an object, with its pose live streamed from Vicon. The object is thrown from one side of a room to another; the humans are free to otherwise move the object. Within its physical capabilities, the robot is capable of recovering its view on the object soon after receiving updates on its position. This reinforces the previous experiment and shows the feasibility of deployment on highly dynamic environments.



% ------- OLD FK DESCRIPTION -------
% We implement a parallelized forward kinematics (FK) function that projects every node in $G$ onto a manifold that represents an end-effector (EE) constraint. This parallel-FK functionality allows the planner to accommodate a wide range of constrained problems; the EE could be a gripper, a camera, or a surgical instrument, and each type of EE may come with different constraints on its position or orientation. For example, it may be beneficial to orient a camera to look at an object, while a scalpel may need to stay perpendicular to a cut surface. 

% We project the camera joint positions for all nodes via parallelized forward kinematics on GPU.  We represent the nodes as one tensor $\mathbf{T}$ of size $\mathbf{N} \cdot \mathbf{k}$, where $\mathbf{N}$ is the number of nodes in $\mathbf{G}$.  Given the camera-to-robot and robot-to-world transformation matrices, and the object of interest's pose in the world frame, we perform simple geometric calculations to project the camera joints.  To simplify the forward kinematics requisite for projection, all states are sampled with the camera parameters set to zero during roadmap construction.
\section{Discussion}
\label{sec:conclusion}

Our experiments show that~\ourplanner~is capable of planning in highly dynamic environments while maintaining contextual-awareness. Table~\ref{tab:planning-results} shows that we achieve an overall speedup factor of more than 130x for a roadmap with 1100 nodes, compared to an equivalent CPU-based PRM. Importantly, this is achieved without sacrificing detection rates or path quality.~\ourplanner~also offers scalable performance: it can replan at up to 75~Hz for a PRM with 500 nodes, or up to 3~Hz for a PRM with 15,000 nodes. Although we use perception-aware planning as a representative example, our method is general in nature and could perform well for other context-aware planning tasks.

Future work will experiment with other neural cost functions beyond those presented in this paper. For example, a privacy-aware function could allow the planner to reliably avoid the privacy-violations and unethical scenarios described in~\cite{shome2023robots}. Alongside experiments with other neural cost functions, we plan to experiment with the use of multiple cost functions and the dynamic weight-adjustment scheme described in~\ref{sec:ncf}. We would also like to inject more parallelism into the various stages of roadmap construction and shift to an approximate-NN search to decrease planning times. Faster planning times will enable the use of denser roadmaps and more complex neural cost functions that could capture richer environmental features.

% Testing this planner on a camera-equipped surgical manipulator could enable the generation of paths that satisfy decoupled EE and perception constraints.

% \input{text/acknowledgement}

%\clearpage
\printbibliography{}

\end{document}

